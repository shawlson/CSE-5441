\documentclass{article}
  
  \usepackage[margin=1in]{geometry}
  \usepackage{amsthm}
  \usepackage{mathtools}
  \pagenumbering{gobble}
  
  \begin{document}
  
  \begin{flushright}
  \textbf{Dan Shawlson} \\
  CSE 5441, T/Th 2:20 P.M.
  \end{flushright}
  
  \begin{Large}
  \centerline{Homework 2}
  \end{Large}
  
  \begin{enumerate}
    \item
    \begin{enumerate}
      \item
      \begin{align*}
      b&=8 & s&=8 & t&=16 & m&=32 \\
      B&=256 & S&=256 & E&=4 & C&=262,144
      \end{align*}
      \item
      Hit rate $x$ = hit rate $y$ = hit rate $a$ = hit rate $b$ = ${}^{31}/_{32}$
      \item
      ${}^{31}/_{32}$
      \item
      Set 0: \{x[0-31], y[0-31], ?, ?\} \\
      Set 1 - 127: ? \\
      Set 128: \{a[0-31], b[0-31], ?, ?\}
      \item
      Set 0: \{x[0-31], y[0-31], ?, ?\} \\
      Set 1: \{x[32-63], y[32-63], ?, ?\} \\
      Set 2: \{x[64-95], y[64-95], ?, ?\} \\
      \ldots \\
      Set 7: \{x[224-255], y[224-255], ?, ?\} \\
      Set 8 - 127: ? \\
      Set 128: \{a[0-31], b[0-31], ?, ?\} \\
      Set 129: \{a[32-63], b[32-63], ?, ?\} \\
      Set 130: \{a[64-95], b[64-95], ?, ?\} \\
      \ldots \\
      Set 143: \{a[480-511], b[480-511], ?, ?\} 
    \end{enumerate}
    \item Use the eviction bits as a counter to determine which line will be occupied by the next fetch. Have the counter start
    at $000$ to signify the first line, and on each fetch increment the counter by one until it reaches $111$ for the last line.
    After storing data in the last line, wrap back around to $000$ for the next fetch. This scheme will work best when a cache line
    is only needed for a short time. However, it will work poorly when there is a lot of contention and some cache lines need to
    have priority over others.
    \item
      Scan 1: Look at each line's tag bits to see if the address is in cache \\
      Scan 2: Scan each line's valid bit to see if there are any lines to occupy \\
      Scan 3: If there are no empty lines, scan each line's eviction bits to see which is the least frequently used
    \item An LRU cache will work well when, in any given time period, a program only uses a small subsection of its data.
    An LFU cache will work well when some pieces of data are accessed very frequently, but the majority of
    data is accessed only a few times.
    \item
    \begin{enumerate}
      \item
      \begin{align*}
        b&=8 & s&=0  & m&=32 & t&=24 \\
        B&=256 & S&=1 & E&=256 & C&=65,536       
      \end{align*}
      \item
      Hit rate $x$ = hit rate $y$ = ${}^{31}/_{32}$
      \item
      ${}^{31}/_{32}$
      \item
      Line 0: x[4096-4127] \\
      Line 1: y[4096-4127] \\
      Line 2: x[4128-4159] \\
      Line 3: y[4128-4159] \\
      \ldots \\
      Line 254: x[8160-8191] \\
      Line 255: y[8160-8191]
    \end{enumerate}
    \item Hierarchy A
  \end{enumerate}
  
  \end{document}
